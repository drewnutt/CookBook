\begin{recipe}[source=\url{https://www.cookwithmanali.com/onion-tomato-chutney/}]{Tomato Chutney}
\index{Tomato}\index{Indian}\index{Chutney}
\ingredients[15]{
    \unit[\nicefrac{1}{2}]{tbsp} & oil or ghee \\
    \unit[1.5]{tsp} &  chana dal (split baby chickpeas) \\
    3-4 & garlic cloves \\
    1 & small onion \\
    2-3 & dried red chiles \\
    2 & large tomatoes \\
    \unit[\nicefrac{1}{2}]{tsp} & salt \\
    \unit[\nicefrac{1}{4}]{tsp} & red chili powder \\
                                & \\
    \multicolumn{2}{l}{\large Tempering/Tadka} \\[0.2cm]
    \unit[1]{tsp} & oil or ghee \\
    \unit[\nicefrac{1}{4}]{tsp} & mustard seeds \\
    pinch & hing \\
    6-7 & curry leaves (optional)\\
}
\preparation{%
    \step Heat 1/2 tablespoon of oil in a kadai/pan on medium heat. Once the oil is hot, add the chana dal and cook for around 2-3 minutes or until the dal is golden in color. Remove the dal from pan and set aside.
    \step To the same pan now add garlic, onion and dried red chilies. Increase the amount of red chilies as per your taste. 
    \step Cook for around 3 minutes until the onions are soft. Then add in the tomatoes and salt and stir.
    \step Cook the tomatoes until they are soft and mushy, around 6-7 minutes. Let it cool down a bit and then transfer to a blender along with the roasted chana dal.
    \step Grind to a smooth paste. Add 1/4 teaspoon kashmiri red chili powder while grinding but that's optional. Transfer chutney to a bowl. 
    \step To make the tadka/tempering, heat 1 teaspoon oil in a small pan on medium heat. Once the oil is hot, add the mustard seeds and let them pop. Then add the hing and curry leaves. Saute for a minute until the curry leaves turn little crisp.
    \step Transfer the tempering to the tomato chutney and mix. Enjoy tomato chutney with dosa/idlis or even parathas.
}
\hint{
        You may also add roasted peanuts to this chutney if you'd like.
}
\end{recipe}
