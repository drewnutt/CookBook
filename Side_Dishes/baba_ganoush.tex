\begin{recipe}[source=\url{https://www.themediterraneandish.com/smoky-eggplant-dip-baba-ganoush/}]{Baba Ganoush}
\index{Eggplant}\index{Vegan}\index{Mediterranean}
\ingredients[]{
1 & large eggplant (or 2-3 small) \\
\unit[2]{tbsp} & tahini \\
1-2 & garlic cloves, chopped \\
\unit[1]{tbsp} & lemon juice \\
to taste & salt and pepper\\
\unit[\nicefrac{1}{4}]{tsp} & crushed red pepper \\
\unit[1]{tsp} & sumac, optional
}
\preparation{%
\step Smoke the eggplant. Turn 1 gas burner on medium or high (will depend on the burner). Using a pair of tongs, turn eggplant every 5 minutes or so until the eggplant is completely tender and it’s skin is charred and crispy (about 15 to 20 minutes.) Don’t worry if the eggplant deflates, it’s supposed to. (You can also do this on a gas or charcoal grill over medium-high heat.) Remove from heat and let the eggplant cool.
\step Once eggplant is cool enough to touch, peel the charred crispy skin off. Discard the stem. Transfer eggplant flesh to a colander; let drain for 3 minutes.
\step Transfer eggplant flesh to the bowl of a food processor. Add tahini paste, yogurt, garlic, lemon juice, salt, pepper, crushed red pepper, and sumac (if using). Give it just a couple of pulses to combine (do not over blend, you want to keep it chunky).
\step Transfer the baba ganoush spread to a small bowl. Cover and refrigerate for an hour (if you don’t have the time, try refrigerating for a few minutes to let the flavors meld and the baba ganoush thicken a bit). Just before serving, top the baba ganoush with a sprinkle of sumac, olive oil, toasted pine nuts, parsley leaves, or anything other Mediterranean spice. Enjoy with a side of warm pita bread or pita chips. 
}
\hint{
If you do not have the means to cook the eggplant over an open flame, you can roast the eggplant in the oven at 425 \faren for about 30-40 min. Cut the eggplant in half and make slits in the skin before roasting.
}
\end{recipe}
