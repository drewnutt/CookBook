\begin{recipe}[bakingtime={\unit[30]{min}},portion=\portion{4},source=\url{https://thewoksoflife.com/mexican-rice-recipe/}]{Mexican Rice}
\index{Rice}\index{Mexican}
\ingredients[10]{
\unit[1\nicefrac{1}{2}]{tbps} & oil \\
\unit[1\nicefrac{1}{4}]{cups} & uncooked white rice \\
\unit[1\nicefrac{1}{2}]{cups} & low-sodium chicken or vegetable stock \\
\unit[1]{tbsp} & tomato paste \\
\unit[\nicefrac{1}{2}]{tsp} & onion powder \\
\unit[\nicefrac{1}{2}]{tsp} & garlic powder \\
\unit[\nicefrac{1}{2}]{tsp} & cumin \\
\unit[\nicefrac{1}{4}]{tsp} & chili powder \\
\unit[\nicefrac{1}{4}]{tsp} & black pepper \\
\unit[\nicefrac{1}{4}]{tsp} & salt \\
}
\preparation{
\step Heat 1.5 tablespoons of oil in a deep skillet set over medium-high heat. Add the rice and stir constantly until the rice begins to turn golden brown. The toastier your rice, the tastier it will be.
\step Next, add the chicken stock. The mixture will bubble up, and should be followed immediately by the tomato paste or tomato sauce, onion powder, garlic powder, cumin, chili powder, black pepper, and salt.
\step Bring to a boil, stirring the tomato paste to dissolve it if using, and cover with a tight-fitting lid. Immediately turn the heat down to low and set a timer for 20 minutes.
\step You may want to check on moisture levels while the rice is cooking. If it looks like the rice needs more water, add \nicefrac{1}{4} cup at a time or until you think the rice is cooked well.
\step When the rice is done, fluff it with a fork and serve!
}

\hint{
\begin{itemize}
    \item If you don't have tomato paste, you can replace it with \nicefrac{1}{2} cup of plain tomato sauce.
    \item Feel free to add more of the spices than in the recipe. The first time I made this, I accidentally doubled the amounts, and it still turned out great.
\end{itemize}
}
\end{recipe}
