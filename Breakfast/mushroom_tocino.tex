\begin{recipe}[source=\url{https://thefoodietakesflight.com/vegan-mushroom-tocino/}]{Mushroom Tocino}
        \index{Filipino}\index{Mushrooms}
        \ingredients[10]{%
                \multicolumn{2}{l}{\large Marinade}\\ [0.2cm]
                \unit[3-4]{tbsp} & brown sugar \\
                \unit[2]{cloves} & garlic \\
                \unit[1]{tbsp} & soy sauce \\
                \unit[1]{tbsp} & vinegar \\
                \unit[3]{tbsp} & pineapple juice \\
                \unit[\nicefrac{1}{2}]{tsp} & annato or atsuete powder (optional) \\
                                            & ground black pepper \\
                \multicolumn{2}{l}{\large Tocino}\\ [0.2cm]
                \unit[250]{g} & fresh oyster mushrooms \\
                \unit[1]{tbsp} & neutral oil \\
                pinch & salt \\
        }
        \preparation{%
                \step Mix together all the ingredients for the marinade in a large bowl until the sugar has dissolved. Feel free to adjust to your desired sweetness and acidity.
                \step Mix the mushrooms in the marinate to completely soak and coat in the sauce. Leave the mushrooms to sit for at least 15 minutes. You can also refrigerate these to marinate overnight.
                \step Note: if cooking some sinangag (garlic fried rice), I suggest to cook this in the same pan before cooking the mushrooms. 
                \step Heat a large pan over medium high heat. Once hot, add in the oil. Place the mushrooms along with the marinating liquid into the pan. Leave the mushrooms to cook over medium to medium high heat until the liquid slowly evaporates. The sugars will slowly cook down and start to beautifully coat the mushrooms.
                \step You can leave the mushrooms untouched for 3-4 minutes to lightly brown and char at the bottom before mixing to cook the remaining sides. You can add a pinch of salt to season the mushrooms, if you’d like.
                \step Continue to cook the mushrooms down until all the sauce has been absorbed and the mushrooms have turned shiny and have a thin glaze-like coating from the sugars that have cooked down.
                \step Serve your tocino with some sinangag (garlic fried rice), suka’t sili (vinegar with some chiles and onions), and kamatis (tomatoes) for the perfect hearty Filipino breakfast that can also be enjoyed any time of the day. Enjoy!
        }
        \hint{
                This recipe can be adapted to use other types of mushrooms, seiten, or tofu!
        }
\end{recipe}
