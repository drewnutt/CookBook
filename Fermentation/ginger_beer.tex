\begin{recipe}[source=\href{https://www.youtube.com/watch?v=LqPko6a3Wh4}{Joshua Weissman}]{Ginger Beer}
\index{Drink}
\ingredients[1]{%
\unit[10]{cups} & water (divided) \\
\unit[255]{g} & sugar (divided) \\
\unit[120]{g} & ginger (divided) \\
}
\preparation{%

{\large Start Ginger Bug}
\begin{enumerate}
    \item Finely chop 22 grams of ginger.
    \item Combine 2 cups water, 28 g sugar, and the ginger to a large glass jar. Cover with a cheesecloth and leave it alone
\end{enumerate}{}
{\large Feeding Ginger Bug}
\begin{enumerate}
    \item Every 24 hours add 28 g sugar and 22 grams chopped ginger to the ginger bug. Continue for about 3 days, or until bubbly.
\end{enumerate}{}
{\large Making Ginger Beer}
\begin{enumerate}
    \item Add 2 quarts of water (8 cups), 180 grams sugar, and 54 grams shredded ginger to a large pot. 
    \item Bring to a boil, then let simmer for 7-8 minutes. Turn off heat and let cool all the way to room temperature (this will take a long time)
    \item Strain the boiled ginger liquid and press out all the juices from the ginger into a large bowl (get one with a spout). Add 110 grams of the strained ginger bug and the juice of 3 lemons.
    \item Add the mixture to bottles that can hold pressure (we don't want an explosion of glass shards), make sure to leave about 2 inches of head space.
    \item Every 24 hours burp the bottles so they don't explode. After 7 days, or whenever you are happy, you can store the bottles in the fridge (you don't have to burp them in the fridge).
\end{enumerate}{}
}
\suggestion{You can use the ginger bug to make any juice into a nice fizzy drink, though be careful on the portioning of the ginger bug (i.e. 2 cups of juice needs about 30 grams of ginger bug)}
\end{recipe}
