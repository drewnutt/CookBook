\begin{recipe}[source=\url{https://www.mexicoinmykitchen.com/sopes-recipe/}]{Sopes}
	\index{Mexican}
	\ingredients[]{%
		\unit[1\nicefrac{1}{2}]{cup} & masa harina \\
	}

	\preparation{%
		\step Mix the masa harina with 1 \nicefrac{1}{4} cup warm water and knead until there is a uniform texture. If the dough feels dry, add more water. The consistency should be similar to playdough.
		\step Cover the dough to keep it moist.
		\step Heat a griddle to medium-high heat.
		\step Divide the dough into 10 pieces of the same weight. Lightly roll into balls.
		\step Flatten the balls into a thick tortilla that is about 4-5 inches in diameter using either a tortilla press or a flat glass dish (use a cut plastic bag to avoid cleanup).
		\step Place the tortilla onto the hot griddle. Turn after one minute.
		\step Turn again after 1 minutee
		\step Remove from the griddle and place into a dry kitchen towl.
		\step After cooling for 30-45 seconds (or until cool enough to touch), form the border by pinching the edges with your fingers. Return to the dry towel.
		\step Immediately before serving, heat 4 tablespoons of oil on medium in a skillet.
		\step Place the sopes on the skillet and lightly fry them on either side, about 30 seconds for each side (enough to give them a slight golden color).
		\step Cooked sopes should be transfered to a plate lined with paper towels.
	}

	\hint{Sopes can be topped with refried beans, lettuce, thinly sliced radishes, avacado, pickled vegetables, etc. Be creative.}
\end{recipe}        
