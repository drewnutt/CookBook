\begin{recipe}[
	portion={8-10 burgers},
	source=\href{https://www.foodandwine.com/recipes/cumin-spiced-red-lentil-burgers}{Food and Wine}]
	{Lentil Burgers}\index{Burger}\index{Vegan}
\ingredients[10]{%
\unit[\nicefrac{3}{4}]{cup} & red lentils \\
\unit[1]{small} & onion \\
1\nicefrac{1}{2} & carrots \\ 
\unit[3]{cloves} & garlic \\
\unit[\nicefrac{3}{4}]{tsp} & cumin \\
\unit[\nicefrac{1}{8}]{tsp} & cayenne pepper \\
\unit[\nicefrac{3}{8}]{cup} & bread crumbs \\ 
\unit[\nicefrac{1}{8}]{cup} & parseley \\ 
1 & egg \\
\unit[\nicefrac{1}{2}]{tbsp} & salt \\
\unit[\nicefrac{1}{4}]{tsp} & black pepper \\ 
}
\preparation{
\step In a saucepan, cover the lentils with 2 inches of water and bring to a boil. Let simmer for ~10 minutes or until softened (but not liquidy). Drain afterwards.
\step Chop the onion and carrots into small pieces. Mince the garlic.
\step In a skillet, heat some oil over medium high heat. Once it is nice and steamy add the onions. Cook, stirring frequently, until the onions are golden.
\step Add the carrots and garlic to the skillet and cook until the carrots begin to soften.
\step Stir in the cumin and cayenne pepper, then remove the skillet from the heat.
\step Combine everything in a large bowl and mix until well combined. Then form into about 10 \nicefrac{1}{4} cup burgers.
\step Using the same skillet from before, add enough oil to coat the pan and heat on medium-high heat.
\step Cook the burgers until browned on both side, about 2-3 minutes per side.
}
\hint{You can use matzo meal instead of using bread crumbs.}
\end{recipe}
