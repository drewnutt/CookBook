\begin{recipe}[source=\href{https://www.justonecookbook.com/zaru-soba-cold-soba-noodles/}{Just One Cookbook}]{Zaru Soba (Cold Soba Noodles)}
	\index{Japanese}
	\ingredients[7]{%
        \unit[4]{packets} & soba noodles \\
	\unit[\nicefrac{1}{4}]{cup} & sake \\
	\unit[\nicefrac{9}{16}]{cup} & mirin \\
	\unit[\nicefrac{1}{2}]{cup} & soy sauce \\
	\unit[1]{(1''x1'') piece} & kombu \\
	\unit[1]{cup} & packed bonito flakes \\
	2 & scallions \\
	}

	\preparation{%
		\step First we will make the dipping sauce. Add the sake to a medium saucepan and bring to a boil over medium-high heat. Let the alcohol evaporate for a few seconds.
		\step Add the soy sauce, mirin, kombu, and dried bonito flakes. Bring to a boil and cook on low for 5 minutes.
		\step Let cool completely, then strain the sauce. Can be stored in the fridge for up to 1 month.
		\step Now to cook the soba noodles. Bring a large pot of (unsalted) water to a boil. Add the soba noodles in a circular motion to separate the noodles, then cook to the package's instructions. Stirring once in a while so they do not stick to each other.
		\step Drain the soba noodles into a sieve, making sure to reserve 1-1\nicefrac{1}{2} cups of the cooking water. Then rinse the noodles with cold water to remove any starch.
		\step Shake off the noodles to drain completely, then add to a large bowl of ice water and wait until the noodles are cool.
		\step Chop the green onions and any other toppings (like nori).
		\step To make 4 servings of the dipping sauce, combine \nicefrac{3}{8} cup of the dipping sauce with 1\nicefrac{1}{8} cup of ice cold water (1:3 ratio of dipping sauce to water). Check the taste, adding more water if its too salty or more sauce if its too diluted.
		\step Serve the noodles with a small dish of the diluted dipping sauce, adding wasabi and/or scallions to the sauce. Dip the noodles into the dipping sauce and slurp the noodles up.
		\step When done eating the noodles, add the reserved soba cooking water to the rest of the diluted dipping sauce and enjoy it as a soup broth.
	}

	\hint{
		\begin{itemize}        
			\item \nicefrac{9}{16} cup is \nicefrac{1}{2} cup + 1 tablespoon
			\item This can be enjoyed with scallions, wasabi and shredded nori. This is often served with tempura.
		\end{itemize}        	
}
\end{recipe}        
