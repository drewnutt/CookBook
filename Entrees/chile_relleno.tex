\begin{recipe}[source=Lucy]{Chile Relleno}
	\index{Mexican}\index{Fried}
\ingredients[9]{%
	7-8 & poblano peppers \\
	\unit[1]{large can} & peeled tomatoes \\
	\unit[1-2]{tsp} & salt \\
	\unit[\nicefrac{1}{4}]{tsp} & black pepper \\
	\unit[\nicefrac{1}{2}]{tsp} & oregano \\
	\unit[1]{medium} & onion \\
	\unit[1]{clove} & garlic, minced \\
	\unit[1]{package} & beyond ground beef \\
	4 & eggs \\
	\unit[\nicefrac{1}{2}]{cup} & neutral oil \\
	\unit[\nicefrac{1}{4}]{cup} & flour \\
}
\preparation{
	\step Roast the poblanos over the stove (using the flame or in a pan). Once they are blackened all over, place immediately into a plastic bag (shopping bag is perfect) and then tie off the bag.
	\step Blend the can of peeled tomatoes. Optionally, you can add jalape\~{n}os to the blender as well. Pour the blended tomatoes into a small pot.
	\step Add salt, pepper, and other spices (oregano, cumin, etc.) as you like to the blended tomatoes. Bring up to heat, but do not boil.
	\step Chop onions and put in a medium frying pan on medium-high heat. Add in minced garlic and meat of your choice. Cook until the meat is fully cooked then remove from heat.
	\step Once the poblanos have cooled down, remove them from the bag and peel off the blackened skin and make a small incision to remove most of the insides.
	\step Stuff the poblanos partially with mozzarella (or other white melting cheese), then add the onion-meat mixture, then fill the rest with more mozzarella.
	\step Separate the egg whites from the egg yolks. Combine the egg whites in a large bowl and whisk until you reach soft peaks. Slowly fold in the yolks to the whipped egg whites.
	\step In a large pan, heat enough oil to shallow-fry the chiles on medium-high heat. Place all of the flour in a shallow bowl.
	\step One at a time, dunk a stuffed poblano into the flour then into the egg mixture, then place into the frying oil. Avoid crowding the pan. Flip after about 3-4 minutes, or when slightly browned and crispy, then fry for an additional 2-3 minutes until both sides are well cooked.
	\step Serve over mexican rice with the tomato sauce poured over the top.
}	
\end{recipe}
