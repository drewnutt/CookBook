\begin{recipe}[source=adapted from \href{https://www.youtube.com/watch?v=reC-BN-_VKI}{Adam Ragusea}]{Spanish Omelette}
	\index{Spanish}\index{Vegetarian}\index{Mediterranean}
	\ingredients[2]{%
        \unit[1]{lb} & Waxy potatoes \\
        1 large & onion \\
	6-8 & eggs \\
	\unit[1]{tsp} & salt \\
	}

	\preparation{%
            \step Cut the potatoes into halves or quarters, depending on how large they are. Then cut them into thin, but not paper thin slices.
	    \step Cut the onion into thin quarter moon slices.
	    \step Crack the eggs into a larger mixing bowl and beat them smooth.
	    \step Cook the onions in a thin coating of olive oil over medium heat for 5 minutes, stirring constantly (think caramelizing onions).
	    \step Add the potates and continue stirring until the potatoes are starting to break apart and there is browning on the edges. You can add some water if it seems like the onions are going to get too brown before the potatoes are done.
	    \step Stir the hot vegetables into the eggs for a couple of minutes to help the eggs start cooking. Add the salt.
	    \step Pour into a pre-heated non-stick pan over medium heat. (This is possible with a cast iron, just need to make sure it is very well seasoned)
	    \step Shortly after pouring into the pan, reduce the heat to low and cook until the omelette is \nicefrac{2}{3}-\nicefrac{3}{4} of the way cooked. You can shake the pan to see how much the omelette jiggles.
	    \step Place a plate, slightly larger than pan over your pan and flip the whole assembly together to let the omelette fall onto the plate. Return the pan to the heat and slide the omelette back into the pan.
	    \step Increase the heat to medium and cook until the bottom is as brown as you want it.
	    \step Turn the omelette out onto a clean plate and let cool completely before serving.
	}

	\hint{
		This is the non-traditional method of cooking a Spanish Omelette. The traditional method is much the same, however, the vegetables are cooked in enough oil to submerge them. The vegetables are then strained and the rest of the process is the same as above.
}
\end{recipe}        
