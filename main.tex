\documentclass{article}
\usepackage[utf8]{inputenc}
\usepackage{geometry}
\usepackage{nicefrac}
\usepackage{gensymb}
\usepackage[nowarnings]{xcookybooky}
\usepackage{imakeidx}
\usepackage{float}


\usepackage{hyperref} % this must be the last package that is imported!

\newcommand{\faren}{\degree F }
\renewcommand{\step} % fixing the error with xcookybooky somehow https://tex.stackexchange.com/questions/481698/missing-endcsname-in-package-xcookybooky-texlive
{%
 \stepcounter{step}%shouldn't be in the argument of lettrine
    \lettrine
    [%
        lines=2,
        lhang=0,          % space into margin, value between 0 and 1
        loversize=0.15,   % enlarges the height of the capital
        slope=0em,
        findent=1em,      % gap between capital and intended text
        nindent=0em       % shifts all intended lines, begining with the second line
    ]{\thestep}{}%
}

\geometry{
 letterpaper,
 left=1in,
 top=1in,
 }

\title{Quarantine Cooking}
\author{Drew McNutt and Monica Dayao}
\date{April 2020 (last updated \today)}
\makeindex[intoc]

\begin{document}

\maketitle
\tableofcontents
\newpage
\section{Introduction}
Welcome to our cookbook. It's simply a compilation of recipes we have enjoyed over the past couple of years. Most recipes are vegetarian (because we are). Recipes have been found from all over the internet and from friends and family. If the section headings aren't what you are looking for, we would recommend looking in the index. We have tagged most recipes with relevant tags so you can find exactly what you want.


\begin{center}
    \huge Enjoy!
\end{center}{}

\newpage
\section{Breakfast}
\begin{recipe}[source=Isaac Spiegel]{Shakshuka}
\index{Cast Iron}\index{Mediterranean}\index{Israeli}\index{Kosher!Milk}
\ingredients[11]{%
\unit[4-5]{tbsp} & olive oil \\
1 & pepper \\
1 & onion \\
1 & can of crushed tomatoes \\
\unit[1-2]{tsp} & paprika \\
\unit[1-2]{tsp} & cumin \\
\unit[1-2]{tsp} & cayenne \\
\unit[1-2]{tsp} & salt \\
1 & feta block \\
\unit[2]{cloves} & garlic \\
4-6 & eggs \\
Some & pepper \\
}
\preparation{%
\step Preheat oven to about 350\faren. Heat 4-5 tablespoons of oil in a large cast iron pan.
\step Chop the onions and peppers and saute in the hot pan until dark brown.
\step Add the garlic, paprika, cumin, cayenne and salt. Saute for about 5 more minutes.
\step Add the crushed tomatoes and let simmer. The length of this simmer is up to you, a longer simmer means a chunkier final product.
\step Dig out a small hole, go all the way to the bottom of the pan and crack an egg into it. Repeat this 3-5 more times.
\step Crack some fresh pepper on top of the eggs. Crumble the entire block of feta, covering the entire pan with cheese. 
\step Put in the oven and cook until the eggs have set. Should take about 10 minutes, but keep an eye on them so you can get nice runny eggs.
}
\end{recipe}

\newpage
\section{Curries}
\begin{recipe}{Baingan Masala (Spicy Eggplant Curry)}
\index{Indian}\index{Spicy}\index{Eggplant}
\ingredients[20]{%
plenty of & oil\\
\unit[\nicefrac{1}{4}]{tsp} & garlic powder \\
\unit[\nicefrac{1}{4}]{tsp} & onion powder \\
4 & green chilies \\
\unit[2]{tbsp} (\unit[22]{g}) & ginger \\
\unit[\nicefrac{3}{4}]{tsp} & turmeric powder \\
5 & tomatoes \\
\unit[3]{tsp} & coriander powder \\
\unit[3]{tsp} & kashmiri chili powder\\
\unit[3]{tbsps} & shredded coconut \\
\unit[2]{tsp} (divided)& salt \\
\unit[800]{g} & eggplant \\
\unit[1\nicefrac{1}{2}]{tsp} & cumin seeds\\
\unit[1]{tsp} & mustard seeds\\
3 & bay leaves \\
1 & cinnamon stick \\
\unit[2]{cups} & water \\
\unit[\nicefrac{3}{4}]{tsp} & garam masala \\
\unit[4-5]{sprigs} & cilantro (coriander) leaves \\
}
\preparation{
\step Heat 5-7 tablespoons oil in a large pan on medium-high heat.
\step Chop up the ginger into small pieces. Cut the tomatoes into large pieces.
\step Add the garlic powder, onion powder, green chilies, chopped ginger, turmeric, and tomatoes to the pan. Mix well and saute for 1 minute.
\step Add the coriander powder, chili powder, and shredded coconut. Mix in and saute for 1 minute. Put heat on low and cover for 2 minutes.
\step Add 1 teaspoon of salt and mix well. Remove the pan from the stove and set aside to cool.Once the mixture is cool enough, blend it into a smooth paste.
\step Cut the eggplant up into large cubes (or cut Xs into the bottoms of small eggplants all the way to the stem).
\step Heat deep frying oil on medium to high heat. (Also now is a good time to pressure cook some rice)
\step Gently add the eggplant to boil(Don't splash the oil!).
\step Deep fry for 3-4 minutes, or until soft, stirring after 1 minute and flipping occasionally for even cooking. Set aside on paper towels to drain oil. 
\step Heat 4-5 tablespoons oil on high heat. Add the cumins seeds, mustard seeds, bay leaves, and cinnamon stick. Saute untill fragrant (the mustard seeds popping is normal, but don't let it go too crazy). 
\step Add the blended paste (be careful it will probably splash a bit), and add about 1 teaspoon salt, or to taste and mix well.
\step Stir until the mixture begins to bubble. Lower the heat and cover. Let simmer for about 6 minutes.
\step Add the water to the blender (gotta clean out all of the yummy paste) to rinse it off.
\step Add the blender water to the stove and mix well. Bring the mixture to a boil. Lower the heat a bit and let simmer for 3 minutes with the lid on.
\step Add the fried eggplants and mix well. Set the heat to medium, cover with a lid and cook for 5 minutes.
\step Add the garam masala and cilantro leaves, mix well. Enjoy!
}
\suggestion{ \begin{itemize}
    \item You can add 4 dried kashmiri chilies at the beginning if you want extra heat, but the 4 green chilies are plenty
    \item Instead of the 5 tomatoes you can use 600 grams crushed tomatoes. Then you can avoid cutting up the tomatoes!
\end{itemize}
}
\end{recipe}

\newpage
\section{Side Dishes}
\begin{recipe}{Hummus}
\index{Chickpeas}\index{Vegan}\index{Mediterranean}\index{Israeli}
\ingredients[]{
\unit[\nicefrac{1}{4}]{cup} & lemon juice \\
\unit[\nicefrac{1}{4}]{cup} & tahini \\
\unit[1]{clove} & garlic \\
\unit[2]{tbsp} & olive oil \\
\unit[\nicefrac{1}{2}]{tsp} & cumin \\
\unit[1]{tsp} & salt \\
\unit[250]{g} & cooked chickpeas \\
}
\preparation{%
\step In a food processor combine the tahini and lemon juice. Process for 1 minute, scrape down the sides of the bowl and process for 30 more seconds.
\step Mince the garlic. Add the olive oil, minced garlic, cumin, and \nicefrac{1}{2} teaspoon of salt to the food processor. 
\step Process for 30 seconds, scrape down the sides of the bowl and process again for 30 seconds.
\step Add half of the drained chickpeas and process for 1 minute. 
\step Scrape down the sides of the bowl and add the remaining chickpeas. Process unitl smooth, 1-2 minutes.
\step If the hummus is still thick or chunky, add water a tablespoon at a time while the processor is on. Should only require about 2 to 3 tablespoons of water. 
}
\end{recipe}

\newpage
\begin{recipe}[
source=\url{https://www.themediterraneandish.com/smoky-eggplant-dip-baba-ganoush/}
]
{Baba Ganoush}
\index{Eggplant}\index{Vegan}\index{Mediterranean}
\ingredients[]{
1 & large eggplant (or 2-3 small) \\
\unit[2]{tbsp} & tahini \\
1-2 & garlic cloves, chopped \\
\unit[1]{tbsp} & lemon juice \\
to taste & salt and pepper\\
\unit[\nicefrac{1}{4}]{tsp} & crushed red pepper \\
\unit[1]{tsp} & sumac, optional
}
\preparation{%
\step Smoke the eggplant. Turn 1 gas burner on medium or high (will depend on the burner). Using a pair of tongs, turn eggplant every 5 minutes or so until the eggplant is completely tender and it’s skin is charred and crispy (about 15 to 20 minutes.) Don’t worry if the eggplant deflates, it’s supposed to. (You can also do this on a gas or charcoal grill over medium-high heat.) Remove from heat and let the eggplant cool.
\step Once eggplant is cool enough to touch, peel the charred crispy skin off. Discard the stem. Transfer eggplant flesh to a colander; let drain for 3 minutes.
\step Transfer eggplant flesh to the bowl of a food processor. Add tahini paste, yogurt, garlic, lemon juice, salt, pepper, crushed red pepper, and sumac (if using). Give it just a couple of pulses to combine (do not over blend, you want to keep it chunky).
\step Transfer the baba ganoush spread to a small bowl. Cover and refrigerate for an hour (if you don’t have the time, try refrigerating for a few minutes to let the flavors meld and the baba ganoush thicken a bit). Just before serving, top the baba ganoush with a sprinkle of sumac, olive oil, toasted pine nuts, parsley leaves, or anything other Mediterranean spice. Enjoy with a side of warm pita bread or pita chips. 
}
\hint{
If you do not have the means to cook the eggplant over an open flame, you can roast the eggplant in the oven at 425 \faren for about 30-40 min. Cut the eggplant in half and make slits in the skin before roasting.
}
\end{recipe}

\newpage
\begin{recipe}[%
preparationtime={\unit[40]{min}},
bakingtime={\unit[45]{min}},
portion=\portion{4-6},
source=Mom
]
{Carrot Souffle}
\index{Jewish}
\ingredients{%
\unit[1]{lb.} & carrots\\
\unit[4]{oz.} & butter\\
3 & eggs\\
\unit[\nicefrac{1}{2}]{Cup} & sugar\\
\unit[3]{tablespoon}& flour\\
\unit[1]{teaspoon} & baking powder\\
\unit[1]{teaspoon} & vanilla extract\\
to taste & salt \\}

\preparation{
\step Peel carrots and cook in salted water until well done (about 30 minutes). Drain Carrots.
\step Melt Butter. Preheat Oven to 275\faren
\step Put eggs, melted butter, sugar, flour, baking powder, and vanilla in a blender and blend well. Add carrots and blend until the mixture resembles a milk shake. 
\step Pour into a greased 8x8 inch glass dish and bake at 275\faren\,for 45 minutes.
}
\hint{
You can use a food processor instead of a blender.
}
\end{recipe}{}

\begin{recipe}[source=Mom]{Matzo Ball Soup}
\index{Soup}\index{Jewish}\index{Kosher!Passover}
\ingredients{%
\unit[\nicefrac{3}{4}]{Cup} & Matzo Meal\\
2 & Eggs\\
\unit[3]{tablespoons} & Oil\\
& Garlic Salt \\
& Dill \\
& Ginger\\
& Black Pepper\\
}
\preparation{%
\step Mix all of the ingredients. Let rest in refrigerator for at least 30 minutes.
\step Lightly form the mixture into balls (do not roll the balls). The key is to form a ball without making it too dense
\step Boil in Veggie Stock for 30 minutes}
\end{recipe}{}

\newpage
\begin{recipe}[
bakingtime={\unit[30]{min}},
portion=\portion{4},
source=\url{https://thewoksoflife.com/mexican-rice-recipe/}]
{Mexican Rice}
\index{Rice}\index{Mexican}
\ingredients[10]{
\unit[1\nicefrac{1}{2}]{tbps} & oil \\
\unit[1\nicefrac{1}{4}]{cups} & uncooked white rice \\
\unit[1\nicefrac{1}{2}]{cups} & low-sodium chicken or vegetable stock \\
\unit[1]{tbsp} & tomato paste \\
\unit[\nicefrac{1}{2}]{tsp} & onion powder \\
\unit[\nicefrac{1}{2}]{tsp} & garlic powder \\
\unit[\nicefrac{1}{2}]{tsp} & cumin \\
\unit[\nicefrac{1}{4}]{tsp} & chili powder \\
\unit[\nicefrac{1}{4}]{tsp} & black pepper \\
\unit[\nicefrac{1}{4}]{tsp} & salt \\
}
\preparation{
\step Heat 1.5 tablespoons of oil in a deep skillet set over medium-high heat. Add the rice and stir constantly until the rice begins to turn golden brown. The toastier your rice, the tastier it will be.
\step Next, add the chicken stock. The mixture will bubble up, and should be followed immediately by the tomato paste or tomato sauce, onion powder, garlic powder, cumin, chili powder, black pepper, and salt.
\step Bring to a boil, stirring the tomato paste to dissolve it if using, and cover with a tight-fitting lid. Immediately turn the heat down to low and set a timer for 20 minutes.
\step You may want to check on moisture levels while the rice is cooking. If it looks like the rice needs more water, add \nicefrac{1}{4} cup at a time or until you think the rice is cooked well.
\step When the rice is done, fluff it with a fork and serve!
}

\hint{
\begin{itemize}
    \item If you don't have tomato paste, you can replace it with \nicefrac{1}{2} cup of plain tomato sauce.
    \item Feel free to add more of the spices than in the recipe. The first time I made this, I accidentally doubled the amounts, and it still turned out great.
\end{itemize}
}
\end{recipe}

\newpage
\begin{recipe}{Corn Fritters}
\index{Fried}\index{Corn}
\ingredients[]{%
\unit[190]{g} & flour \\
\unit[2]{cups} & corn \\
\unit[50]{g} & shredded cheddar cheese \\
\unit[15]{g} & chives \\
\unit[1/2]{cup} & milk \\
2 & eggs \\
\unit[1]{tsp} & salt \\
\unit[1]{tsp} & kashmiri chili powder \\
\unit[\nicefrac{1}{2}]{tsp} & onion powder \\
\unit[\nicefrac{1}{2}]{tsp} & garlic powder \\
\unit[\unit[1]{2}]{cup} & oil
}
\preparation{
\step Combine all ingredients (except for the oil) in a large bowl
\step Heat the oil (make sure its enough to coat the entire bottom of the pan) in a skillet on medium heat
\step When the oil is hot, add 2 tablespoons of the mixture to the oil and press down to make flat
\step Flip after about 2 minutes, or until the bottom is sufficiently crispy
\step Fry on the last side for another 2 minutes, then remove from the oil and place on a paper towel lined plate
}
\hint{You can replace the egg with a chia seed egg and the milk with a non-dairy milk
}
\end{recipe}

\newpage
\section{Entrees}
\begin{recipe}[
portion={8-10 burgers},source=\url{https://www.foodandwine.com/recipes/cumin-spiced-red-lentil-burgers}]{Lentil Burgers}
	\index{Burgers}
\ingredients{%
\unit[\nicefrac{3}{4}]{cup} & red lentils \\
\unit[1]{small} & onion \\
1\nicefrac{1}{2} & carrots \\ 
\unit[3]{cloves} & garlic \\
\unit[\nicefrac{3}{4}]{tsp} & cumin \\
\unit[\nicefrac{1}{8}]{tsp} & cayenne pepper \\
\unit[\nicefrac{3}{8}]{cup} & bread crumbs \\ 
\unit[\nicefrac{1}{8}]{cup} & parseley \\ 
1 & egg \\
\unit[\nicefrac{1}{2}]{tbsp} & salt \\
\unit[\nicefrac{1}{4}]{tsp} & black pepper \\ 
}
\preparation{
\step In a saucepan, cover the lentils with 2 inches of water and bring to a boil. Let simmer for ~10 minutes or until softened (but not liquidy). Drain afterwards.
\step Chop the onion and carrots into small pieces. Mince the garlic.
\step In a skillet, heat some oil over medium high heat. Once it is nice and steamy add the onions. Cook, stirring frequently, until the onions are golden.
\step Add the carrots and garlic to the skillet and cook until the carrots begin to soften.
\step Stir in the cumin and cayenne pepper, then remove the skillet from the heat.
\step Combine everything in a large bowl and mix until well combined. Then form into about 10 \nicefrac{1}{4} cup burgers.
\step Using the same skillet from before, add enough oil to coat the pan and heat on medium-high heat.
\step Cook the burgers until browned on both side, about 2-3 minutes per side.
}
\hint{You can use matzo meal instead of using bread crumbs.}
\end{recipe}

\newpage
\begin{recipe}
[
preparationtime={\unit[15]{minutes}},
bakingtime={\unit[20]{minutes}},
source=\url{https://www.mexicoinmykitchen.com/pasta-with-creamy-roasted-poblano-sauce/}]{Creamy Roasted Poblano Pasta}
\index{Mexican}\index{Pasta}\index{Vegetarian}
	\ingredients{%
		\unit[1]{package} & spaghetti or other pasta\\
		\unit[6]{large} & poblano peppers \\
		\unit[1]{cup} & onion \\
		\unit[2]{cloves} & garlic \\
		\unit[1]{cup} & milk \\
		\unit[1\nicefrac{1}{2}]{cup} & heavy cream \\
		\unit[4]{tsp} & bouillon \\
		\unit[1\nicefrac{1}{2}]{cup} & corn \\
	}
	\preparation{%
		\step Roast the poblano peppers (can be done over the gas stove or in a pan). Then peel and devein the peppers (peeling is made easier when they are more charred).
		\step Cook the pasta according to the package instructions. Drain and set aside until the sauce is ready.
		\step Heat 1 tablespoon of oil in a skillet over medium heat, add the onion. Cook for about 2 minutes then add the garlic. Continue cooking until the onion is transparent.
		\step Place the roasted poblanos, cream, milk and bouillon into a blender. Blend until smooth (Note: if using heavy cream then it may form curds in the blender, but these will melt when warmed later)
		\step Add the sauce to the skillet and cook on low for about 6 minutes.At this point you can add the corn and any other veggies. Stir frequently.
		\step Combine the pasta and sauce. Enjoy!
	}
	\hint{
		Instead of heavy cream you can use mexican cream, sour cream, or cream cheese. You just need some sort of cream.
	}
\end{recipe}
\newpage
\section{Bread}
\begin{recipe}[
preparationtime={\unit[2.5]{hours}},
bakingtime={\unit[30-35]{minutes}},
portion={2 loaves},
source=\url{https://www.tasteofhome.com/recipes/basic-homemade-bread/}]{White Sandwich Bread}
\index{Bread}
\ingredients[8]{%
\unit[1]{cup} & water \\
\unit[2]{tsp} & active dry yeast \\
\unit[1]{cup} & milk \\
\unit[2]{tbsp} & butter \\
\unit[2]{tbsp} & sugar \\
\unit[1]{tbsp} & salt \\
\unit[715]{g} & all-purpose flour \\
& oil \\
}
\preparation{%
\step Mix the yeast, sugar and warm water(warm but should be able to keep a finger in it for a bit) in the bowl of the stand mixer. Let stand for about 5 minutes
\step Melt the butter and combine with the milk and salt. Add the mixture to the yeast mixture. Add 130 grams (1 cup) of the flour and mix until a clumpy mixture is formed.
\step Add the remaining 585 grams of flour to the stand mixture, gradually, mixing as you add. Mix unitl a shaggy, floury dough is formed.
\step Knead the dough for 8 to 10 minutes. Add flour if the dough is sticking to the sides of the bowl a tablespoon at a time. Dough should form a ball without sagging, and spring back when poked.
\step Grease a large bowl and transfer the dough to the greased bowl. Flip the dough ball so it is covered in oil. Cover and let rise in a warm place until doubled in size, about one hour.
\step Split the dough into two equal pieces and form each piece into a loose ball. Let rest for 10 minutes.
\step Grease 2 loaf pans. Smash a ball into a rectangle with the palms of your hand. Then fold the rectangle into overlapping thirds, pinch to close the sides and the ends. Fold in half by pressing down in the middle and bringing the sides together, pinch to close on the ends and side. Repeat for the second loaf.
\step Move the loaves to the loaf pans, by flipping them seam side down. Then let rest in a warm place until they start to dome over the sides of the pan, about an hour.
\step Heat the oven to 425 \faren. Slash the top of the loaves with a serated knife before you put them in the oven. When you add the loaves to the oven reduce the heat to 375\faren. Bake for 30 to 35 minutes. They should sound hollow when tapped.
\step Cool on a wire rack until completely cool before slicing.
}
\end{recipe}
\begin{recipe}[
preparationtime={\unit[2.5-4]{hrs}, depending on rise time},
bakingtime={\unit[15]{min}},
portion={8 4-5 inch buns},
source=\url{https://smittenkitchen.com/2009/07/light-brioche-burger-buns/}]{Light Brioche Burger Buns}
\index{Bread}\index{Burger}
\ingredients[12]{
\unit[1]{cup} + \unit[1]{tbsp} & water \\
\unit[3]{tbsp} & warm milk \\
\unit[2]{tsp} & active dry yeast \\
\unit[2\nicefrac{1}{2}]{tbsp} & sugar \\
2 & eggs \\
\unit[3]{cups} & bread flour \\
\unit[\nicefrac{1}{3}]{cup} & all-purpose flour \\
\unit[1\nicefrac{1}{2}]{tsp} & salt\\
\unit[2\nicefrac{1}{2}]{tbsp} & unsalted butter, softened\\
& sesame seeds (optional)
}
\preparation{
\step In a measuring cup, combine 1 cup warm water, milk, yeast, and sugar. Let stand until foamy, about 5 minutes. Meanwhile, beat 1 egg.
\step In a large bowl, whisk the flours with salt. Add the butter, divided into small pieces, and rub into the flower between your fingers, making crumbs.
\step Stir in yeast mixture and beaten egg and knead until smooth and elastic, 8-10 minutes. The dough will be on the sticky side, so it can be a bit messy. Try and leave the dough tackier than you would a round loaf.
\step Shape dough into a ball and return it to the bowl. Cover the bowl with plastic wrap or a towel and let rise in a warm place until doubled, 1-2 hours.
\step Line a baking sheet with parchment paper or a silicone baking mat. Divide the dough into 8 equal parts. Gently roll each into a ball and arrange 2-3 inches apart on the baking sheet. Cover loosely with plastic wrap or towel and let buns rise in a warm place for 1-2 hours.
\step Set a large shallow pan of water on the oven floor. Preheat oven to 400\faren with rack in center. Beat remaining egg with 1 tbsp water and brush on top of buns. Sprinkle with sesame seeds.
\step Bake buns, turning sheet halfway through, for about 15 minutes, until tops are golden brown. Transfer to a rack to cool completely.
}
\end{recipe}

\newpage
\begin{recipe}[
preparationtime={\unit[20]{min}},
bakingtime={\unit[20-25]{min}},
portion={8 tortillas},
source=\url{https://www.kingarthurflour.com/recipes/simple-tortillas-recipe}]{Flour Tortillas}
\index{Tortilla}\index{Mexican}\index{No Yeast}
\ingredients[9]{
\unit[2\nicefrac{1}{2}]{cups} (298g) & all-purpose flour \\
\unit[1]{tsp} & baking powder \\
\unit[\nicefrac{1}{2}]{tsp} & salt \\
\unit[\nicefrac{1}{4}]{cup of any} & lard (57g), butter (57g), shortening (48g), or vegetable oil (50g)\\
\unit[\nicefrac{3}{4}-1]{cup} & hot water
}
\preparation{
\step In a medium-sized bowl, whisk together the flour, baking powder, and salt.
\step Add the lard (or butter, or shortening; if you're using vegetable oil, add it in step 3). Use your fingers or a pastry blender to work the fat into the flour until it disappears. Coating most of the flour with fat inhibits gluten formation, making the tortillas easier to roll out.
\step Pour in the lesser amount of hot water (plus the oil, if you're using it), and stir briskly with a fork or whisk to bring the dough together into a shaggy mass. Stir in additional water as needed to bring the dough together.
\step Turn the dough out onto a lightly floured counter and knead briefly, just until the dough forms a ball. If the dough is very sticky, gradually add a bit more flour.
\step Divide the dough into 8 pieces. Round the pieces into balls, flatten slightly, and allow them to rest, covered, for about 30 minutes (see tips, below). If you wish, coat each ball lightly in oil before covering; this ensures the dough doesn't dry out (not necessary though).
\step While the dough rests, preheat an ungreased cast iron griddle or skillet over medium high heat, about 400\faren.
\step Working with one piece of dough at a time, roll into a round about 8" in diameter. Keep the remaining dough covered while you work. Fry the tortilla in the ungreased pan for about 30 seconds on each side, or until air bubble form and browning occurs on each side. Repeat with the remaining dough balls.
}
\hint{
Use just enough water to bring the dough together. Too much water and you'll end up with a sticky dough that is difficult to handle.
}

\end{recipe}

\newpage
\begin{recipe}[
preparationtime={\unit[15]{min} + \unit[20-30]{min} rest},
bakingtime={\unit[15]{min}},
portion={12 rotis},
source=\url{https://www.cookwithmanali.com/roti-recipe/}]{Roti/Chapati}
\index{Indian}\index{Bread}\index{No Yeast}
\ingredients[7]{
\unit[2]{cups}(270g) + \unit[\nicefrac{1}{4}]{cup} & atta (whole wheat flour) \\
\unit[1-2]{tsp} & oil, optional\\
\unit[around \nicefrac{3}{4}]{cup} & water \\
some & ghee
}
\preparation{
\step Take 2 cups (270 grams) atta in a large bowl. You can add little oil if you like here, but this is completely optional.
\step Start adding water, little by little. As you add water, mix with your hands and bring the dough together. You may need more or less water depending on the kind of flour.
\step Once the dough comes together, start kneading the dough. Knead with the knuckles of your finger, applying pressure.
\step Fold the dough using your palms and knead again applying pressure with your knuckles. Keep kneading until the dough feels soft and pliable. If it feels hard/tight, add little water and knead again. If it feels too sticky/soft, add some dry flour and mix.
\step Once done, the dough should be smooth. Press the dough with your fingers, it should leave an impression.
\step Cover the dough with a damp cloth or paper towel for 20 to 30 minutes.
\step After the dough has rested, give it a quick knead again. Divide the dough into 12 equal parts, each weighing around 35 to 37 grams.
\step Start working with one ball, while keep the remaining dough balls covered with a damp cloth so that they don't dry out.
\step Take one of the balls and press it between your fingers to make it smooth. Then roll it between your palms to make it round and smooth. There should be no cracks. Press the round dough ball and flatten it slightly.
\step Now take around 1/4 cup atta in a plate for dusting the roti while rolling it. Dip the prepared dough ball into the dry flour and dust it from all sides.
\step Then start rolling the roti, using a rolling board and rolling pin. Roll it thin until you have a 5 to 6 inch diameter circular roti. You will have to dip the roti in dry flour several times while rolling the roti. Anytime the dough starts sticking to the rolling pin, dip the roti into the atta from both sides and then continue rolling.
\step Heat the tawa (skillet) on medium-high heat. Make sure the tawa is hot enough before you place the roti on the tawa. Dust excess flour off the rolled roti and place it on the hot tawa.
\step Let it cook for 15-30 seconds until you see some bubbles on top side. At this point flip the roti, you don't want the first side to cook too much.
\step Now, let the other side cook more than the first side, around 30 seconds more. Use a tong to see how much it has cooked from the second side now. If you see brown spots all over, means it has cooked enough.
\step Now, remove the roti from the tawa using a tong and place it directly on flame with the first side (which was little less cooked) directly on the flame. The roti if rolled evenly will puff up, flip with a tong to cook the other side as well. The roti is done when it has brown spots, don't burn it.
\step Apply ghee on the rotis immediately. Make all the roti similarly. Serve warm.
}
\hint{
If you have an induction stovetop, you can still make these. Instead of placing the roti directly on the flame, flip back onto the hot tawa (first side on bottom) and press roti with a paper towel, cotton cloth, or spatula. It will puff up.
}
\end{recipe}

\begin{recipe}[
preparationtime={\unit[15]{min}},
portion={2 pizzas},
source=\url{https://www.bonappetit.com/recipe/pizza-dough-2}]{Pizza Dough}
\index{Pizza}
\ingredients[6]{
\unit[\nicefrac{3}{4}]{cup} & warm water \\
\unit[2\nicefrac{1}{4}]{tsp} & dry active yeast \\
\unit[2]{cups} (260g) & all purpose flour \\
\unit[1]{tsp} & sugar\\
\unit[\nicefrac{3}{4}]{tsp} & salt\\
\unit[3]{tbsp} & vegetable oil
}
\preparation{
\step Pour 3/4 cup warm water into small bowl; stir in yeast. Let stand until yeast dissolves, about 5 minutes.
\step Mix 2 cups flour, sugar, and salt in mixer.
\step Add yeast mixture and 3 tablespoons oil; process until dough forms a sticky ball.
\step Transfer to lightly floured surface. Knead dough until smooth, adding more flour by tablespoonfuls if dough is very sticky, about 1 minute. You can also do this in the mixer.
\step Transfer to a bowl coated in oil; turn dough in bowl to coat with oil. Cover bowl and let dough rise in warm draft-free area until doubled in volume, about 1 hour.
\step Punch down dough. Roll out dough. Start in center of dough, working outward toward edges but not rolling over them.
}
\hint{
This recipe can be done a day ahead. Leave the dough in the fridge after kneading and take it out a couple of hours before to let rise before rolling out.
}

\end{recipe}
\newpage
\begin{recipe}[
preparationtime={\unit[30]{min} + \unit[45]{min} rise},
bakingtime={\unit[15]{min}},
portion={16 breadsticks},
source=\href{https://www.foodnetwork.com/recipes/food-network-kitchen/almost-famous-breadsticks-recipe-1972945}{`Almost-Famous Breadsticks' from Food Network Magazine}]{Simple Breadsticks}
\index{Bread}
\ingredients[10]{
\unit[\nicefrac{1}{4} + 1\nicefrac{1}{4}]{cup} & warm water \\
\unit[2\nicefrac{1}{4}]{tsp} & dry active yeast \\
\unit[4\nicefrac{1}{4}]{cups} (560g) & all purpose flour \\
\unit[2]{tbsp} & sugar\\
\unit[1]{tbsp} & salt\\
\unit[2]{tbsp} & unsalted butter, softened\\
\unit[3]{tbsp} & unsalted butter, melted\\
\unit[\nicefrac{1}{2}]{tsp} & kosher salt \\
\unit{pinch} & garlic powder \\
\unit{pinch} & dried oregano
}
\preparation{
\step Place 1/4 cup warm water in the bowl of a mixer; sprinkle in the yeast and set aside until foamy, about 5 minutes. 
\step Add the flour, butter, sugar, fine salt and 1\nicefrac{1}{4} cups plus 2 tablespoons warm water; mix with the paddle attachment until a slightly sticky dough forms, 5 minutes. 
\step Knead the dough by hand on a floured surface until very smooth and soft, 3 minutes.
\step Roll into a 2-foot-long log; cut into 16 1\nicefrac{1}{2}-inch-long pieces.
\step Knead each piece slightly and shape into a 7-inch-long breadstick; arrange 2 inches apart on a parchment-lined baking sheet. Cover with a cloth; let rise in a warm spot until almost doubled, about 45 minutes. 
\step Preheat the oven to 400 \faren. Make the topping: Brush the breadsticks with 1\nicefrac{1}{2} tablespoons of the butter and sprinkle with \nicefrac{1}{4} teaspoon kosher salt. 
\step Bake until lightly golden, about 15 minutes. Meanwhile, combine the remaining \nicefrac{1}{4} teaspoon salt with the garlic powder and oregano.
\step Brush the warm breadsticks with the remaining 1\nicefrac{1}{2} tablespoons melted butter and sprinkle with the flavored salt.
}

\end{recipe}

\newpage
\section{Dessert}
\begin{recipe}[source=\url{https://www.southernliving.com/recipes/fresh-peach-cobbler}]{Peach Cobbler}
\index{Cobbler}\index{Fruit}
\ingredients[9]{%
\unit[1/4]{cup} & butter \\
\unit[65]{g} & flour \\ 
\unit[200]{g} & sugar, divided \\
\unit[\nicefrac{1}{2}]{tbsp} & baking powder \\
\unit[1]{pinch} & salt \\
\unit[\nicefrac{1}{2}]{cup} & milk \\
\unit[2]{cups} & peach slices \\
\unit[\nicefrac{1}{2}]{tbsp} & lemon juice \\
\unit[1]{pinch} & ground cinnamon \\
\unit[1]{pinch} & ground nutmeg \\
}
\preparation{%
\step Preheat oven to 375 \faren. Add the butter to an 8x8 baking dish and put into oven to melt.
\step Combine the flour, 100 g of sugar, baking powder, and salt Add the milk and stir until the dry ingredients are moistened.
\step Bring the remaining sugar, peaches, and lemon juice up to a boil over high heat while stirring constantly.
\step Pour the batter over the melted butter, do not stir. Then pour the peach mixture over the batter, again without stirring. Sprinkle with cinnamon and nutmeg (optional)
\step Bake for 40 to 45 minutes.
}
\end{recipe}

\newpage
\begin{recipe}[source=\url{https://sallysbakingaddiction.com/super-moist-carrot-cake/}]{Carrot Cake}
\index{Cake}
\ingredients[20]{%
\unit[200]{g} & brown sugar \\
\unit[\nicefrac{3}{4}]{cup} & vegetable oil \\
\unit[60]{g} & greek yogurt \\
3 & eggs \\
\unit[2]{tsp} & vanilla extract \\
\unit[250]{g} & all-purpose flour \\
\unit[1]{tsp} & baking soda \\
\unit[2]{tsp} & ground cinnamon\\
\unit[\nicefrac{1}{4}]{tsp} & ground nutmeg \\
\unit[\nicefrac{1}{2}]{tsp} & salt \\
\unit[2]{cups} or \unit[260]{g} & finely grated carrots\\
\unit[\nicefrac{3}{4}]{cup} & pecan pieces\\
\\
\unit[8]{oz} (224g) & cream cheese, room temp\\
\unit[\nicefrac{1}{2}]{cup} (115g) & unsalted butter, softened\\
\unit[240-300]{g} & confectioners' suger\\
\unit[2]{tbsp} & heavy cream\\
\unit[2]{tsp} & vanilla extract\\
& salt to taste
}
\preparation{
\step Preheat oven to 350 \faren (177 \degree C). Spray 9 or 10inch springform pan with nonstick cooking spray.
\step Set out the cream cheese for the frosting so it may soften as you make the cake batter.
\step In a large bowl with a handheld or stand mixer fitted with a paddle attachment on medium speed, combine the brown sugar and oil. Beat in the yogurt until fully combined – about 60 seconds. Mixture will be gritty and thick. 
\step Add the eggs, one at a time, beating well after each addition. Mix in the vanilla. Set aside.
\step In a separate bowl, combine the flour, baking soda, cinnamon, nutmeg, and salt. 
\step With a spatula, manually stir the dry ingredients into the wet ingredients until just combined and all flour pockets are gone – do not overmix. Fold in the finely shredded carrots and pecan pieces. Pour or spoon batter into prepared springform pan.
\step Bake cake for 32-38 minutes or until toothpick inserted in the center comes out clean. Do not overbake, which will dry out cake. Check the cake at 30 minutes, then again at 32. Allow cake to cool completely before frosting.
\step To make the frosting, beat the softened cream cheese and butter together on medium speed for 2-3 minutes until soft, creamy, and combined thoroughly.
\step Add 2 cups of powdered sugar and beat until thick and combined. Add 2 Tablespoons heavy cream and 2 teaspoons vanilla extract. Beat on medium speed for 2 more minutes.
\step Add more powdered sugar until desired thickness is reached. Add salt to taste. 
}
\hint{
This recipe also can be made into 12 cupcakes. Bake time is 17-18 minutes.
}
\end{recipe}

\newpage
\begin{recipe}{Peanut Butter Cookies}
\index{Easy}
\ingredients[2]{
\unit[1]{cup} & sugar \\
\unit[1]{cup} & peanut butter \\
1 & egg \\
}
\preparation{
\step Preheat the oven to 350\faren
\step Cream together the peanut butter and sugar
\step Add the egg and mix until well combined. 
\step Portion onto a baking sheet into spoonfuls. Using a fork, press down on the cookies twice making a crosshatch.
\step Bake for about 8 minutes, let it rest on a rack for about 5-10 minutes.
}

\hint{
\begin{itemize}
    \item You can add chocolate chips or oats to make these even more delicious!
    \item You can also bake these in a cast iron pan. In this case, bake for about 10-12 minutes.
\end{itemize}}
\end{recipe}{}

\begin{recipe}[
preparationtime={\unit[15]{min}},
bakingtime={\unit[30]{min}},
source=\url{https://www.thekitchenismyplayground.com/2015/07/chocolate-crack-pie.html}]
{Chocolate Crack Pie}
\index{Easy}\index{Pie}\index{Chocolate}
\ingredients[5]{%
\unit[4]{oz.} & dark chocolate \\
\unit[\nicefrac{1}{2}]{cup} & butter \\
\unit[1]{cup} & sugar \\
2 & eggs \\
1 & pie crust
}
\preparation{%
\step Preheat oven to 350\faren
\step Melt the butter and chocolate together over low heat in a small saucepan
\step Remove from heat once melted and well combined. Add the sugar to the mixture and mix well.
\step Let the mixture rest for about 3 minutes. Beat the eggs in a small bowl.
\step Add the chocolate mixture to the eggs and mix well. Transfer to pie crust.
\step Bake the pie in the oven for about 30 minutes. Let cool for about 1 hour.
}
\end{recipe}{}

\begin{recipe}[
preparationtime={\unit[10]{min}},
bakingtime={\unit[20]{min}},
source=\url{https://www.makingthymeforhealth.com/vegan-avocado-brownies/}]{Vegan Avocado Brownies}
\index{Brownie}\index{Vegan}\index{Chocolate}\index{Avocado}
\ingredients[]{
\nicefrac{1}{2} & medium ripe avocado \\
\unit[1]{cup} + \unit[2]{tbsp} & plant-based milk \\
\unit[\nicefrac{1}{4}]{cup} & maple syrup \\
\unit[\nicefrac{1}{2}]{cup} & sugar \\
\unit[1]{cup} & all-purpose flour \\
\unit[\nicefrac{1}{2}]{cup} & unsweetened cocoa powder \\
\unit[1]{tsp} & baking soda \\
\unit[\nicefrac{1}{2}]{tsp} & salt \\
\unit[\nicefrac{1}{2}]{cup} & chocolate chips
}
\preparation{
\step Preheat the oven to 350\faren. Lightly grease an 8x8" baking dish.
\step In a blender, combine the avocado, soymilk, maple syrup and sugar. Blend for about 15-20 seconds, until smooth.
\step In a large bowl, combine the flour, cocoa powder, baking soda, and salt then stir together. 
\step Pour the wet ingredients in the blender into the bowl with the dry. Stir until combined. Then add the chocolate chips and mix until they are evenly distributed.
\step Bake in the oven for 15-20 minutes, until set. You should be able to stick a fork in the center and have it come out clean.
\step Allow to cool for at least 15 minutes before serving.
}

\hint{
\begin{itemize}
    \item You can substitute the flour for whole-wheat flour if desired. 
    \item Feel free to add walnuts or any other type of nut else to this recipe!
\end{itemize}
}
\end{recipe}

\newpage
\begin{recipe}{Chickpea Cookie Dough}
\ingredients[]{%
\unit[250]{g} & cooked chickpeas \\
\unit[\nicefrac{1}{4}]{cup} & nut/seed butter \\
\unit[1]{tsp} & vanilla extract \\
\unit[\nicefrac{1}{4}]{cup} & Almond flour \\
\unit[2]{tbsp} & maple syrup \\
\unit[\nicefrac{1}{4}]{cup} & chocolate chips \\
\unit[\nicefrac{1}{2}]{tsp} & salt \\
}
\preparation{
\step Add everything to a food processor except for the chocolate chips, process until smooth
\step Add the chocolate chips and mix them in by hand.
\step Cool for at least 2 hours or so in the fridge, overnight is usually best
}
\end{recipe}

\newpage
\section{Fermentation}
\begin{recipe}[source=\url{https://www.youtube.com/watch?v=LqPko6a3Wh4}]{Ginger Beer}
\index{Drink}
\ingredients[1]{%
\unit[10]{cups} & water (divided) \\
\unit[255]{g} & sugar (divided) \\
\unit[120]{g} & ginger (divided) \\
}
\preparation{%

{\large Start Ginger Bug}
\begin{enumerate}
    \item Finely chop 22 grams of ginger.
    \item Combine 2 cups water, 28 g sugar, and the ginger to a large glass jar. Cover with a cheesecloth and leave it alone
\end{enumerate}{}
{\large Feeding Ginger Bug}
\begin{enumerate}
    \item Every 24 hours add 28 g sugar and 22 grams chopped ginger to the ginger bug. Continue for about 3 days, or until bubbly.
\end{enumerate}{}
{\large Making Ginger Beer}
\begin{enumerate}
    \item Add 2 quarts of water (8 cups), 180 grams sugar, and 54 grams shredded ginger to a large pot. 
    \item Bring to a boil, then let simmer for 7-8 minutes. Turn off heat and let cool all the way to room temperature (this will take a long time)
    \item Strain the boiled ginger liquid and press out all the juices from the ginger into a large bowl (get one with a spout). Add 110 grams of the strained ginger bug and the juice of 3 lemons.
    \item Add the mixture to bottles that can hold pressure (we don't want an explosion of glass shards), make sure to leave about 2 inches of head space.
    \item Every 24 hours burp the bottles so they don't explode. After 7 days, or whenever you are happy, you can store the bottles in the fridge (you don't have to burp them in the fridge).
\end{enumerate}{}
}
\suggestion{You can use the ginger bug to make any juice into a nice fizzy drink, though be careful on the portioning of the ginger bug (i.e. 2 cups of juice needs about 30 grams of ginger bug)}
\end{recipe}

\section{Instant Pot}
\subsection*{Basic recipes and Durations}
\begin{table}[H]
    \centering
    \begin{tabular}{c|c|c|c|c}
        Food Item & Food Amount & Water Amount & Pressure Duration (min) & Natural Release(min) \\ \hline
        Basmati Rice & \unit[1]{cup} & \unit[1]{cup} & 8 & 2\\ \hline
        Brown Rice & \unit[1]{cup} & \unit[1]{cup} & 15 & 5 \\ \hline
        Chickpeas & \unit[1]{lb.} & \unit[6]{cup} & 50 & 10 \\ \hline
        Blackbeans & \unit[1]{lb.} & \unit[6]{cup} & 30 & 15 \\
    \end{tabular}
    \label{tab:InstantPot}
\end{table}

\begin{recipe}[source=\url{https://www.simplyhappyfoodie.com/instant-pot-black-eyed-peas/}]{Black-Eyed Peas}
\ingredients[]{%
\unit[1]{tbsp} & olive oil \\
\unit[1]{small} & onion \\
1 & bell pepper \\
4-5 & mushrooms \\
\unit[\nicefrac{1}{2}]{tsp} & thyme \\
\unit[3]{tsp} & paprika \\
\unit[\nicefrac{1}{2}]{tsp} & black pepper \\
\unit[1]{tsp} & salt \\
\unit[4]{cloves} & garlic \\
3 & dried chilies \\
\unit[3\nicefrac{1}{2}]{cups} & vegetable broth \\
\unit[2]{tsp} & balsamic vinegar \\
\unit[250]{g} & black-eyed peas \\
}
\preparation{%
\step Turn the pressure cooker to the saute function and add the oil.
\step Dice the onion, bell pepper and mushrooms. When the pot is hot add them to the pot, stirring occasionally until the onions start to turn translucent.
\step Add the thyme, paprika, pepper, and salt. Stir.
\step Add the garlic and dried peppers. Cook for about 30 seconds, stirring frequently.
\step Add the broth, vinegar, and black-eyed peas to the pot. Stir well.
\step Place the lid on the pressure cooker and set to pressure cook for 17 minutes (when the cooking is done start making the rice).
\step Let naturally vent for 15 more minutes.
}
\end{recipe}

\section{Substitutes}
\begin{recipe}{Chia Seed Egg}
\index{Egg}\index{Vegan}\index{Vegetarian}
\ingredients[]{%
\unit[2]{tbsp}(\unit[2]{tsp}) & Chia Seeds (Ground)\\
\unit[3]{tbsp} & Boiling Water \\
}
\preparation{
\step Mix together the chia seeds and boiling water
\step Wait 5 minutes for the chia seeds to gel, then use as you would a normal egg (may need to wait for the "egg" to cool first)
}
\end{recipe}

\printindex

\end{document}
