\begin{recipe}[preparationtime={\unit[20]{min}},bakingtime={\unit[20-25]{min}},portion={8 tortillas},source=\url{https://www.kingarthurflour.com/recipes/simple-tortillas-recipe}]{Flour Tortillas}
\index{Tortilla}\index{Mexican}\index{No Yeast}
\ingredients[9]{
\unit[2\nicefrac{1}{2}]{cups} (298g) & all-purpose flour \\
\unit[1]{tsp} & baking powder \\
\unit[\nicefrac{1}{2}]{tsp} & salt \\
\unit[\nicefrac{1}{4}]{cup of any} & lard (57g), butter (57g), shortening (48g), or vegetable oil (50g)\\
\unit[\nicefrac{3}{4}-1]{cup} & hot water
}
\preparation{
\step In a medium-sized bowl, whisk together the flour, baking powder, and salt.
\step Add the lard (or butter, or shortening; if you're using vegetable oil, add it in step 3). Use your fingers or a pastry blender to work the fat into the flour until it disappears. Coating most of the flour with fat inhibits gluten formation, making the tortillas easier to roll out.
\step Pour in the lesser amount of hot water (plus the oil, if you're using it), and stir briskly with a fork or whisk to bring the dough together into a shaggy mass. Stir in additional water as needed to bring the dough together.
\step Turn the dough out onto a lightly floured counter and knead briefly, just until the dough forms a ball. If the dough is very sticky, gradually add a bit more flour.
\step Divide the dough into 8 pieces. Round the pieces into balls, flatten slightly, and allow them to rest, covered, for about 30 minutes (see tips, below). If you wish, coat each ball lightly in oil before covering; this ensures the dough doesn't dry out (not necessary though).
\step While the dough rests, preheat an ungreased cast iron griddle or skillet over medium high heat, about 400\faren.
\step Working with one piece of dough at a time, roll into a round about 8" in diameter. Keep the remaining dough covered while you work. Fry the tortilla in the ungreased pan for about 30 seconds on each side, or until air bubble form and browning occurs on each side. Repeat with the remaining dough balls.
}
\hint{
Use just enough water to bring the dough together. Too much water and you'll end up with a sticky dough that is difficult to handle.
}

\end{recipe}
