\begin{recipe}[source=\href{https://cooking.nytimes.com/recipes/1016071-homemade-pita-bread}{New York Times Cooking}]{Pita}
	\index{Bread}\index{Mediterranean}\index{Israeli}	
	\ingredients[4]{%
		\unit[2]{tsp} & active dry yeast \\
		\unit[\nicefrac{1}{2}]{tsp} & sugar \\
		\unit[35]{g} & whole wheat flour \\
		\unit[260]{g} & all-purpose flour \\
		% Original recipe said to use 310 g and reserve 1/2 cup for dusting, so may need slightly more or less flour 
		\unit[1]{tsp} & kosher salt \\
		\unit[2]{tbsp} & olive oil \\
	}

	\preparation{%
		\step In a large bowl, combine the yeast and sugar with 1 cup of lukewarm water. Stir until well dissolved.
		\step Add the whole-wheat flour and about 40 grams of the all-purpose flour. Whisk everything together and place in a warm area, uncovered for about 15 minutes or until frothy and bubbling.
		\step Add the salt, olive oil, and the remaining all-purpose flour. Stir until the mixture forms a shaggy mass. Dust with a little bit of flour and knead in the bowl for 1 minute.
		\step Turn dough onto work surface and knead for about 2 minutes or until smooth. Cover and let rest for 10 minutes.
		\step Knead again for 2 minutes. Try not too add too much flour, the dough should be soft and a bit moist. (After this step you can put the dough into a container and refridgerate up-to overnight, making sure to bring back to room temperature before continuing)
		\step Put the dough in a clean bowl, cover, and let rest in a warm place until doubled in size (about 1 hour)
		\step Heat the oven to 475 \faren. Put a baking sheet (or large cast iron) on the bottom shelf to heat up.
		\step Punch down the dough and divide into 8 pieces. Form each piece into a small ball, cover the balls, and let rest for 10 minutes.
		\step Press a ball into a flat disk with a rolling pin. Then roll into a 6" circle, then into an 8" circle (about \nicefrac{1}{8}" thick)
		\step Place the dough onto the hot baking sheet and put back into the oven. After 2 minutes, flip the dough (it should be puffed). Bake for 1 more minute and transfer to a towel so that the bread stays soft.
		\step Repeat steps 9 and 10 for the remaining 7 pieces.
	}
            
\end{recipe}        
