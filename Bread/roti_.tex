\begin{recipe}[preparationtime={\unit[15]{min} + \unit[20-30]{min} rest},bakingtime={\unit[15]{min}},portion={12 rotis},source=\url{https://www.cookwithmanali.com/roti-recipe/}]{Roti (Chapati)}
\index{Indian}\index{Bread}\index{No Yeast}
\ingredients[7]{
\unit[2]{cups}(270g) + \unit[\nicefrac{1}{4}]{cup} & atta (whole wheat flour) \\
\unit[1-2]{tsp} & oil, optional\\
\unit[around \nicefrac{3}{4}]{cup} & water \\
some & ghee
}
\preparation{
\step Take 2 cups (270 grams) atta in a large bowl. You can add little oil if you like here, but this is completely optional.
\step Start adding water, little by little. As you add water, mix with your hands and bring the dough together. You may need more or less water depending on the kind of flour.
\step Once the dough comes together, start kneading the dough. Knead with the knuckles of your finger, applying pressure.
\step Fold the dough using your palms and knead again applying pressure with your knuckles. Keep kneading until the dough feels soft and pliable. If it feels hard/tight, add little water and knead again. If it feels too sticky/soft, add some dry flour and mix.
\step Once done, the dough should be smooth. Press the dough with your fingers, it should leave an impression.
\step Cover the dough with a damp cloth or paper towel for 20 to 30 minutes.
\step After the dough has rested, give it a quick knead again. Divide the dough into 12 equal parts, each weighing around 35 to 37 grams.
\step Start working with one ball, while keep the remaining dough balls covered with a damp cloth so that they don't dry out.
\step Take one of the balls and press it between your fingers to make it smooth. Then roll it between your palms to make it round and smooth. There should be no cracks. Press the round dough ball and flatten it slightly.
\step Now take around 1/4 cup atta in a plate for dusting the roti while rolling it. Dip the prepared dough ball into the dry flour and dust it from all sides.
\step Then start rolling the roti, using a rolling board and rolling pin. Roll it thin until you have a 5 to 6 inch diameter circular roti. You will have to dip the roti in dry flour several times while rolling the roti. Anytime the dough starts sticking to the rolling pin, dip the roti into the atta from both sides and then continue rolling.
\step Heat the tawa (skillet) on medium-high heat. Make sure the tawa is hot enough before you place the roti on the tawa. Dust excess flour off the rolled roti and place it on the hot tawa.
\step Let it cook for 15-30 seconds until you see some bubbles on top side. At this point flip the roti, you don't want the first side to cook too much.
\step Now, let the other side cook more than the first side, around 30 seconds more. Use a tong to see how much it has cooked from the second side now. If you see brown spots all over, means it has cooked enough.
\step Now, remove the roti from the tawa using a tong and place it directly on flame with the first side (which was little less cooked) directly on the flame. The roti if rolled evenly will puff up, flip with a tong to cook the other side as well. The roti is done when it has brown spots, don't burn it.
\step Apply ghee on the rotis immediately. Make all the roti similarly. Serve warm.
}
\hint{
If you have an induction stovetop, you can still make these. Instead of placing the roti directly on the flame, flip back onto the hot tawa (first side on bottom) and press roti with a paper towel, cotton cloth, or spatula. It will puff up.
}
\end{recipe}
