\begin{recipe}[source=\url{https://www.cookwithmanali.com/sarson-ka-saag/},portion=\portion{4-6}]{Sarson ka Saag Paneer}
        \index{Indian}\index{Paneer}
\ingredients[28]{
        \multicolumn{2}{l}{\large To Pressure Cook}\\[0.2cm]
        \unit[250]{g} & mustard greens \\
        \unit[250]{g} & mixed greens such as spinach \\
        1 & medium onion, chopped\\
        5-6 & garlic cloves, chopped \\
        \unit[1]{in} & ginger, chopped\\
        2 & green chilies (or to taste), chopped \\
        \unit[3]{in} & white radish, chopped \\
        \unit[\nicefrac{1}{2}]{tsp} & red chili powder \\
        \unit[1]{tsp} & salt \\
        \unit[1.5]{cups} & water\\
                         & \\
        \unit[2]{tbsp} & maize flour \\
                       & \\
        \multicolumn{2}{l}{\large Tempering/Tadka} \\[0.2cm]
        \unit[2]{tbsp} & ghee \\
        \unit[\nicefrac{1}{4}]{tsp} & hing/asafetida \\
        3-4 & garlic cloves, chopped \\
        1 & medium onion, chopped \\
        2 & dried red chilies \\
        \unit[\nicefrac{1}{2}]{tsp} & coriander powder \\
        \unit[\nicefrac{1}{2}]{tsp} & garam masala \\
                                    & \\
        \multicolumn{2}{l}{\large Paneer} \\[0.2cm]
        \unit[2]{blocks} & paneer, cut into half in cubes \\
        \unit[2]{tbsp} & ghee or oil
}
\preparation{
\step Add the chopped onion, garlic, ginger and green chilies to the pressure cooker on `saute' mode. Saute for a few minutes until parts of the onion turn brown. Wash and chop the greens then add them to the pressure cooker.
\step Add tomatoes and white radish. Then add the red chili powder and salt. Add 1.5 cups water and stir.
\step Cook on high pressure for 5 mins and then let the pressure release naturally. Alternatively, you can also cook everything on a stove top for 20-25 minutes until soft.
\step Open the instant pot and then use an immersion blender to puree the saag. If you don't have an immersion blender, wait for it cool down a bit and then puree using your regular blender.
\step Blend to a coarse paste. You can make it as smooth/coarse as you like.
\step Transfer saag to another pot on stove top over medium-low heat. Add 2 tablespoons of maize flour to the saag and mix, this helps in thickening the saag. Regular flour or cornstarch can also be used if you don't have maize flour.
\step Set heat to low and let the saag simmer for 20 to 25 minutes on low heat. It will thicken as it simmers.
\step In the meantime, cut up your paneer into half inch cubes and pan fry with oil or ghee until they are brown on each side.
\step For the final tadka, heat a small pan on medium heat. Once the pan is hot add ghee to it and then add hing and chopped garlic cloves. Saute for few seconds and then add the chopped onion and dried red chilies.
\step Cook until the onions and garlic turn light golden brown. Add the coriander powder and garam masala and mix.
\step Transfer the tadka/tempering to the saag and mix. Then transfer the paneer into the saag and mix.
\step Serve sarson ka saag with makki roti, sliced onion, jaggery and white butter! 
}
\hint{ 
        If you have an air fryer, you can use it to fry up the paneer cubes.
}
\end{recipe}
