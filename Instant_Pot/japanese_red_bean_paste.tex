\begin{recipe}[source=\url{https://www.justonecookbook.com/pressure-cooker-anko-red-bean-paste/}]{Japanese Red Bean Paste (Anko)}
\ingredients[5]{%
\unit[250]{g} & azuki beans \\
\unit[1000]{ml} & water (1:4 bean:water) \\
\unit[250]{g} & sugar \\
\unit[\nicefrac{1}{8}]{tsp} & salt 
}
\preparation{%
        \step Put the 250 g azuki beans in a strainer and place it inside a large bowl. Rinse the azuki beans in running water until water is clear. Discard any pieces that are floating. Drain water.
        \step Transfer the beans to the Instant Pot and add 1000 ml of water to your pressure cooker. 
        \step Turn the Instant Pot on and cook with high pressure for 25 minutes. Naturally release the pressure for 15-20 minutes after cooking. 
        \step Scoop the foam on the surface and discard (if you prefer the more refined taste, optional). Pick one bean and mash it with your fingers. If it is mashed easily, it's done. 
        \step Drain the azuki beans through a fine sieve. 
        \step Put azuki beans back in the Instant pot and add the sugar. Press the `Saute' button and select the `Low' option.
        \step Let the sugar dissolve completely, stirring occasionally with a wooden spoon. Continue cooking until you can draw a line in the azuki bean mixture with the wooden spatula and see the bottom of the pot for 1 second. Then turn off the Instant Pot and take out the inner bowl from the Instant Pot and \textbf{let the mixture cool for 5-10 minutes}. The mixture will thicken more as it cools down. 
        \step Transfer the warm azuki beans into the food processor or blender. This may need to be done in batches (original recipe uses 14 cup food processor). Alternatively, you can use a very fine mesh strainer and press the mixture through with a wooden spoon.
        \step Run the food processor or blender until the mixture becomes smooth texture.
        \step Transfer to an airtight container. When it’s cooled and thickened more, it’s ready to use.
}
\hint{
        This can be kept in the freezer. For easier storage, divide into 100g portions.
}
\end{recipe}
